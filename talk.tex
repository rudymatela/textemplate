\RequirePackage{ifjobname}
\ifjobname{talk-handout}{
	\documentclass[14pt,handout]{beamer}
}{
	\documentclass[14pt]{beamer} % default is 11pt
}
% Font sizes:
%   default:      11pt
%   larger:       14pt
%   power point:  17pt  (too large for my taste)

\usetheme{default}
\usecolortheme{dove}

\usepackage[utf8]{inputenc}
\usepackage{graphicx}
\usepackage{verbatim}
\usepackage{moreverb}
\usepackage{multicol}
\usepackage{macros}
\usepackage{beamer-helper}
\usepackage{clistings}
\usepackage[T1]{fontenc}
\setbeamertemplate{itemize items}[square]

% Font -- choose one
\usepackage{gillius}
%\usepackage{comicneue} % [nosfdefault] to not set as default
%\usepackage{raleway}

% TODO: Add logos on title page?
% TODO: Optional tikz gradient background?
%\usepackage{tikz}
%\usebackgroundtemplate{
%  \begin{tikzpicture}
%    \path [top color = green!10, bottom color = white]
%      (0,0) rectangle (\paperwidth,.25\paperheight);
%  \end{tikzpicture}}

\title[Template]{\LaTeX{} talk template}
\author[John Doe]{%
	John Doe \\
	\texttt{\footnotesize johndoe@example.com}
}
\institute[DSS -- SUS]{%
	Department of Scientific Science \\
	Scientific University of Someplace
}
\date{\today}



% If you want to reshow the outline at the beginning of each section
%\AtBeginSection[] {
%	\framet[Outline]{
%		\tableofcontents[currentsection,hideothersubsections]
%	}
%}

\newcommand{\colorize}[1]{%
\setbeamercolor{normal text}{bg=#1!6}%
\setbeamercolor{structure}{fg=#1!24!black}% ,bg=#1!16}
}
\setbeamercovered{transparent=12, again covered={\opaqueness<1->{24}}}


\begin{document}

\frameabs{
	I find it useful to put the paper/talk abstract
		as the first slide.
	People that arrived earlier can read the abstract
		while people enter the meeting room.
	This slide looks better justified.
	The abstract will possibly have to be shortened for insertion here.
	Or, this text can be prepended by \texttt{\textbackslash small}
		so more can be fit into the abstract slide.
	\textbf{Warning:
		this template makes heavy use of custom macros.}
	Consult the beamer documentation
		if you want to typeset
		in the ``standard way''.
}


\frame{\titlepage}

\framet[Outline]{\tableofcontents}


\colorize{white}
\sframeT{Typesetting lists}

\framei[Bullet points]{
	\item Use \texttt{\textbackslash framei} or \texttt{\textbackslash ize}
		to typeset bullet points
	\item It is possible to typeset sub-items: \ize{
		\item like
		\item this
	}
	\item That's it
}

\framei{
	\pitem This is an animated slide on \texttt{talk.pdf}
	\pitem This wont be shown animated on \texttt{talk-handout.pdf}
	\pitem Slide title is optional
}

\framed{
	\item[What?] A description
	\item[How?] Using \texttt{\textbackslash framed}
	               or \texttt{\textbackslash don}
	\item[When?] Now!
}

\framee{
	\item The first item
	\item The second item \enu[i.]{
		\item Inner item
		\item Another
	}
	\item The third item
}


\colorize{blue}
\sframeT{Typesetting definitions, theorems or examples}

\begin{frame}[<+>]
\begin{definition}
Definitions are defined by stating definitions iff:
the definition is definable.

(this slide is animated)
\end{definition}

% TODO: Fix color of example
\begin{example}[definition]
If \texttt{a = b} then \texttt{a = b}
\end{example}
\end{frame}



\colorize{green}
\sframeT{Typesetting images and graphics}

\framei[Image and bullets]{
	\item An important graph follows:
	\incg[0.8]{euler}
	\item indeed...
}

\framet[Centered image, no bullets]{
	\cincg[0.8]{diagram}
}


\colorize{cyan}
\sframeT{Typesetting text}

\framet[Text-only slide]{
	A text-only slide.

	Lorem ipsum dolor sit amet, consectetur adipisicing elit, sed do eiusmod
	tempor incididunt ut labore et dolore magna aliqua. Ut enim ad minim
	veniam, quis nostrud exercitation ullamco laboris nisi ut aliquip ex ea
	commodo consequat.
}

\framej[Justified text slide]{
	A justified text slide.

	Lorem ipsum dolor sit amet, consectetur adipisicing elit, sed do eiusmod
	tempor incididunt ut labore et dolore magna aliqua. Ut enim ad minim
	veniam, quis nostrud exercitation ullamco laboris nisi ut aliquip ex ea
	commodo consequat.
}

\colorize{orange}
\sframeT{Controlling appearance in handout}

% Specifying whether a slide is going to appear in the handout is not supported
% by custom sty macros.  Explicit beamer code looks like this:
\begin{frame}<1|handout:0>{Slide not in handout}
	\begin{itemize}
		\item This slide:
		\begin{itemize}
			\item is not in the handout
		\end{itemize}
	\end{itemize}
\end{frame}

\begin{frame}<0|handout:1>{Slide only handout}
	\begin{itemize}
		\item This slide:
		\begin{itemize}
			\item is only in the handout
		\end{itemize}
	\end{itemize}
\end{frame}

\colorize{red}
\sframeT{Typesetting source code listings}
% To use verbatim or listing environments you have to make the frame fragile.
% This makes the compilation a bit slower, potentially a lot slower if you do
% this for a lot of slides.
\begin{frame}[fragile,t]{Hello world}
	\begin{verbatim}
	int main()
	{
	    printf("Hello world");
	    return 0;
	}
	\end{verbatim}
\end{frame}

\framett[Hello world, typeset with no verbatim]{
	/* Typesetting this may be better \\
    ~* because fragile frames \\
    ~* are slow to compile */ \\
	int main() \\
	\{ \\
	~~~~printf("Hello world"); \\
	~~~~return 0; \\
	\} \\
}

\begin{frame}[fragile,t]{Hello world, via \texttt{(c)listings}}
\begin{haskell}
hwords :: [String]
hwords = [ "Hello"
         , "world" ]

hw :: String
hw = unwords hwords

main :: IO ()
main = putStrLn hw
\end{haskell}
\end{frame}

\colorize{white}
\frameT{Thank you!}

\end{document}

\RequirePackage{ifjobname}
\ifjobname{talk-handout}{
	\documentclass[14pt,handout]{beamer}
}{
	\documentclass[14pt]{beamer} % default is 11pt
}

\usetheme{default}
\usecolortheme{dove}

\usepackage[utf8]{inputenc}
\usepackage{graphicx}
\usepackage{verbatim}
\usepackage{moreverb}
\usepackage{multicol}
\usepackage{macros}
\usepackage{beamer-helper}
\usepackage[T1]{fontenc}
\setbeamertemplate{itemize items}[square]

% Font -- choose one
\usepackage{gillius}
%\usepackage{comicneue} % [nosfdefault] to allow inclusion
%\usepackage{raleway}

\title[My talk]{
	\LaTeX\ talk template
}
\author[John Doe]{%
	John Doe \\
	\texttt{\footnotesize johndoe@example.com}
}
\institute[DSS -- SUS]{%
	Department of Scientific Science \\
	Scientific University of Someplace
}
\date{\today}

% If you want to reshow the outline at the beginning of each section
%\AtBeginSection[] {
%	\framet[Outline]{
%		\tableofcontents[currentsection,hideothersubsections]
%	}
%}

\newcommand{\colorize}[1]{
\setbeamercolor{normal text}{bg=#1!4}
\setbeamercolor{structure}{fg=#1!24!black} % ,bg=#1!16}
}

\begin{document}

\frame{\titlepage}

\framet[Outline]{ \tableofcontents }


\colorize{white}
\sframeT{Simple examples}

\framei[Bullet points]{
	\item This shows how to typeset bullet points
	\item It is possible to typeset sub-items: \ize{
		\item like
		\item this
	}
	\item That's it
}

\framei[Image and bullets]{
	\item An important graph follows:
	\incg[0.8]{euler}
	\item indeed...
}

\framet[Centered image, no bullets]{
	\cincg{diagram}
}

\framei[Animating]{
	\pitem This is an:
	\pitem animated
	\pitem slide
}

\framet[Text-only slide]{
	A text-only slide.
	Lorem ipsum dolor sit amet, consectetur adipisicing elit, sed do eiusmod
	tempor incididunt ut labore et dolore magna aliqua. Ut enim ad minim
	veniam, quis nostrud exercitation ullamco laboris nisi ut aliquip ex ea
	commodo consequat.
}

\colorize{orange}
\sframeT{More simple examples}

\framed[Item descriptions]{
	\item[This] is like this
	\item[That] is like that
	\item[Something] is like a thing
}

% Specifying whether a slide is going to appear in the handout is not supported
% by custom sty macros.  Explicit beamer code looks like this:
\begin{frame}<1|handout:0>{Slide not in handout}
	\begin{itemize}
		\item This slide:
		\begin{itemize}
			\item is not in the handout
		\end{itemize}
	\end{itemize}
\end{frame}

\begin{frame}<0|handout:1>{Slide only handout}
	\begin{itemize}
		\item This slide:
		\begin{itemize}
			\item is only in the handout
		\end{itemize}
	\end{itemize}
\end{frame}

% To use verbatim or listing environments you have to make the frame fragile.
% This makes the compilation a bit slower, potentially a lot slower if you do
% this for a lot of slides.
\begin{frame}[fragile,t]{Hello world}
	\begin{verbatim}
	int main()
	{
	    printf("Hello world");
	    return 0;
	}
	\end{verbatim}
\end{frame}

\colorize{white}
\frameT{Thank you!}

\end{document}

% LaTeX document template
%
% This uses several custom commands defined in sty/macros.sty
\documentclass[12pt,a4paper]{article}
\usepackage[utf8]{inputenc}
\usepackage[english]{babel}
\usepackage{verbatim}
\usepackage{a4wide}

% Custom made packages (see sty/ folder):
\usepackage{clistings}
\usepackage{macros}

% Bibliography
\usepackage{natbib}
\bibliographystyle{plainnat}  % also: cell, plain or alpha

% Link stuff
\usepackage{hyperref}
\definecolor{citation}{RGB}{0,96,0}
\hypersetup{colorlinks=true, linkcolor=black, citecolor=citation, filecolor=black, urlcolor=blue}

% For Brazilian Portuguese:
% \usepackage[brazil]{babel}
% \usepackage{indentfirst}

% For sans serif:
% \renewcommand{\familydefault}{\sfdefault}
% \usepackage{sfmath}

\title{\LaTeX\ Document Template}
\author{
	John Doe \\
	{\small University of Someplace} \\
	\texttt{\footnotesize johndoe@example.com} \\
\and
	Jane Doe \\
	{\small University of Someplace} \\
	\texttt{\footnotesize janedoe@example.org} \\
}
%\date{\today} % noop, change this to customize


\begin{document}


\maketitle

\begin{abstract}
This is a \LaTeX\ document template.  Instead of creating documents from
scratch, this document can be used as a starting point.  It makes significant
use of custom shorthand macros (check source for details).
\end{abstract}

\tableofcontents
\pagebreak


\section{Introduction}

This is a \LaTeX\ document template.  Instead of creating documents from
scratch, this document can be used as a starting point.  The source tex makes
significant use of custom shorthand macros (check source for details).  It
shows how to typeset items (\S\ref{sec:items}), marked paragraphs
(\S\ref{sec:paragraphs}), source code (\S\ref{sec:source}), figures and tables
(\S\ref{sec:ft}).

According to \citet{proposing-proposals} it is important to cite or reference
stuff, either as part of the text itself or at the end of paragraphs
\citep{something,proposing-proposals}.

This will be shown on output, obviously.
\pass{This will also be shown.}
\cmm{But this will not.}


\section{Items}
\label{sec:items}

\subsection{Descriptions and Itemizations}

Below is an example of descriptions and itemizations.  They are typeset into
two columns. \mc{
	\ize{
		\item wooo;
		\item hoooooo.
	}
	\don{
		\item[Aspect A:] aspect A.
		\item[Aspect B:] aspect B.
	}
}


\subsection{Enumeration}
\label{sec:enumeration}

Three kinds of enumeration follow: numerals, letters and roman numerals.
They are disposed in three columns.

\mc[3]{
	\enu{
		\item first item;
		\item second item;
		\item third item;
	}
	\enu[a)]{
		\item first item;
		\item second item;
		\item third item;
	}
	\enu[i.]{
		\item first item;
		\item second item;
		\item third item;
	}
}


\subsubsection{A sub-sub-section}

This is a sub-sub-section.


\section{Marked Paragraphs}
\label{sec:paragraphs}

\paragraph{First paragraph} This is the text of the first paragraph.

\paragraph{Second paragraph} This is the text of the second paragraph.


\section{Source code}
\label{sec:source}

A Haskell program follows:

\begin{haskell}
	function :: [a] -> [a]
	function [] = []            -- comment
	function (x:xs) = xs ++ [x]
\end{haskell}


\section{Figures and Tables}
\label{sec:ft}

Figure \ref{fig:graph} shows a graph.
Table \ref{tab:table} has all letters in the alphabet.

\fig[.35]{graph}{A directed graph}

\begin{stable}{clr}{table}{Alphabet}
	\hline
    The  &  Quick  &  Brown  \\
    \hline
    Fox  &  Jumps  &  Over   \\
    The  &  Lazy   &  Dog    \\
	\hline 
\end{stable} 


\bibliography{bibliography}


\end{document}

